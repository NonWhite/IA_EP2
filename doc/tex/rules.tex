\section{Regras}
\label{sec:rules}

Além das características que descrevem cada personagem, a base de conhecimento precisa de regras para poder inferir qual é a personagem que o usuário está pensando. Nesta seção serão explicadas as regras colocadas na base de conhecimento.

\subsection{Regras de descrição}
\label{subsec:description}
Para que o programa consiga inferir o personagem, tem que existir fatos diferentes na base de conhecimento para cada um. Então para cada característica o jeito de representar um fato é da seguinte forma:
	\begin{itemize}
		\item $V$\_type
		\item $V$\_color
		\item $V$\_size
		\item $V$\_weight
	\end{itemize}
Onde $V$ é um dos possíveis valores para cada atributo. Além disso, para as características binarias (true,false) só vão existir se fossem verdadeiros para algum personagem.\\
Por exemplo, a primeira personagem se representa
	\begin{lstlisting}
		bulbasaur :- verify( grass_type ) ,
					verify( posion_type ) ,
					verify( green_color ) ,
					verify( small_size ) ,
					verify( light_weight ) ,
					verify( has_evolution ) ,
					verify( is_starter ) , ! .
	\end{lstlisting}
Por último, a função ${verify}$ é usada para perguntar ao usuário por aquele fato.

\subsection{Regras de exclusão}
\label{subsec:exclusion}

